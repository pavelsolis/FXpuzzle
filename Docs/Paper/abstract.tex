\begin{abstract}
	This paper argues that the null or weak response of emerging market currencies to domestic monetary policy documented in the literature is the result of wide event windows. An event study with intraday data for Mexico shows that an unanticipated tightening appreciates the currency and flattens the yield curve, consistent with the evidence for advanced economies. With daily event windows, however, only the yield curve responds to monetary policy. Noise in daily exchange rate returns explains the lack of response of the currency. Such noise gives rise to a bias that declines after controlling for potential omitted variables.
	
	\vspace{.5cm}

	\noindent \textit{Keywords}: Monetary policy, exchange rate, yield curve, emerging markets, high-frequency data, event study.
	
	\noindent \textit{JEL Classification}: E52, E58, E43, F31, G14. 
	\vfill
	
	\pagebreak
\end{abstract}